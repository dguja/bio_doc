\documentclass[times, utf8, seminar, numeric]{fer}

\usepackage{booktabs}
\usepackage{hyperref}
\usepackage{enumitem}
\usepackage{mathtools}
\usepackage{listings}
\usepackage{multirow}
\usepackage{tikz}
\usepackage{svg}
\usepackage{graphicx}
\usepackage[section]{placeins}
\usepackage[croatian, algoruled, noend, lined, linesnumbered, algochapter]{algorithm2e}

\graphicspath{{./figures/}}

\newcommand\todo[1]{\textbf{\textcolor{red}{TODO: #1}}}

\begin{document}

\title{essaMEM: finding maximal exact matches using enhanced sparse suffix arrays}
\author{Daniel Guja, Antun Razum, Petra Rebernjak}
\voditelj{doc. dr. sc. Mirjana Domazet-Lošo}

\maketitle
\tableofcontents

\chapter{Uvod}
Komparativna genomika je područje bioinformatike u kojem se značajke genoma različitih organizama uspoređuju. U ovoj grani bioinformatike, uspoređuju se dijelovi ili pak cijeli genomi kako bi se pronašle i proučavale sličnosti i razlike. Glavni princip komparativne genomike je taj da su zajedničke značajke organizama zapisane u DNK te su one evolucijski sačuvane.

Početak komparativne genomike je vrijeme kada su postali dostupni cijeli genomi dvaju organizama. To su genomi bakterija \textit{Haemophilus influenzae} i \textit{Mycoplasma genitalium} 1995. godine. Komparativna genomika je danas standardni alat u analizi svih novih slijedova genoma. Ovo područje bioinformatike je otkrilo velike sličnosti između nekih organizama kao što su ljudi i čimpanze. Također, pokazala je i velike razlike u sastavu gena u različitim evolucijskim lozama.

Rastom sve većeg broja dostupnih genoma javlja se potreba za novim tehnologijama u usporedbi nizova. Kombiniranjem i slaganjem genoma pomoću tih tehnologija i usporedba dobivenih usko povezanih genoma biti će vrlo značajna u otkrivanju novih znanja, od otkrivanja gena i velikih genomskih reorganizacija za utvrđivanje razlika između srodnih vrsta do otkrivanja novih lijekova za do sada neizlječive bolesti.

Jedan od alata bitnih za daljnji razvoj je izračunavanje maksimalnih točnih slaganja između dvaju nizova. Izračunavanje maksimalnih točnih slaganja između nizova bitno je za sljedeće: izračunavanje i rangiranje sličnosti genoma te utvrđivanja ključnih točaka u uspoređivanju genom-genom.

Cilj ovog projektnog zadatka je proučiti \textit{essaMEM}, složeni algoritam i strukture podataka za efikasno računanje maksimalnih točnih slaganja za velike nizove te implementirati podskup funkcionalnosti koje su prožili autori u radu \cite{essa}.

\chapter{essaMEM}

\textit{essaMEM} je vrlo složeni algoritam za pronalaženje maksimalnih točnih slaganja. Maksimalno točno slaganje \engl{Maximal Exact Matches, MEM} je niz koji predstavlja točno slaganje između dva niza koji se ne može produžiti lijevo niti desno bez da se dogodi neslaganje između dva niza koja se uspoređuju. Za pronalaženje MEM-ova potrebna je pogodna struktura podataka. Jedna od njih je sufiksno stablo. Sufiksno stablo je osnovna struktura podataka koja omogućuje brzu analizu nizova. Sufiksno stablo ima široku primjenu u bioinformatici. Mnoštvo primjena opisano je posebno početkom 21. stoljeća. Kako je potreba za korištenjem sufiksnih stabala rasla, bilo je potrebno pronaći novu strukturu podataka koja će unaprijediti već postojeću. Tako se došlo do alternativne strukture podataka, sufiksnog polja \engl{suffix array, SA}. Sufiksno polje je zamijelo sufiksno stablo. Glavni razlog tome je memorijski utrošak sufiksnog stabla. Sufiksno stablo za ulazni niz od $n$ znakova zahtijeva najmanje $10n$ okteta memorijskog prostora u praksi, a često i $15n-20n$ okteta. Sufiksno polje teorijski zauzima $O(n log n)$ bita, a u praksi obično $4n$ okteta.

Uobičajeno se koriste sufiksna stabla ili unaprijeđena sufiksna polja \engl{Enhanced Suffix Arrays, ESA} za pronalaženje MEM-ova. ESA se sastoji od četiri polja: sufiksnog polja, polja najvećih zajedničkih prefiksa \engl{Longest Common Prefix, LCP}, child polja i polja sufiksnih poveznica \engl{suffix link arrays} koja sadrže dijelove informacija iz sufiksnog stabla te zajedno dostižu potpunu ekspresivnost sufiksnih stabala. Dakle, svaki problem koji se može riješiti korištenjem sufiksnih stabala, može se riješiti i korištenjem ESA s istom asimptotskom složenošću.

Khan \textit{et. al.} u \cite{ssa} savjetuje korištenje rijetkih sufiksnih polja \engl{Sparse Suffix Arrays, SSA}. Kod SSA, indeksira se svaki \textit{K}-ti sufiks. Parametar \textit{K} naziva se parametar rijetkosti, \engl{sparsness factor}. Algoritam koji koriste autori u radu \cite{ssa} za pronalaženje MEM-ova temelji se na SSA i pronalazi MEM-ove puno brže od prethodnih metoda. Uz brže pronalaženje MEM-ova algoritam koristi i značajno manje memorije od prethodnih metoda. Posljedično, uporabom SSA moguće je indeksirati znatno veće genome.

Vyverman \textit{et. al.} u \cite{essa} želi poboljšati metodu predstavljenu u \cite{ssa} implementiranjem rijetkih child polja za velike parametre rijetkosti. Dodatno, korištenjem sufiksnih poveznica \engl{suffix links} i child polja gradi se unaprijeđeno rijetko sufiksno polje \engl{Enhanced Sparse Suffix Array, ESSA} koje ima istu ekspresivnost kao i sufiksno stablo za podnizove veće od \textit{K}. Memorijski učinkoviti algoritmi za pronalaženje MEM-ova mogu se podijeliti u dvije kategorije: \textit{online} i \textit{index} metode. essaMEM pripada kategoriji indeksnih metoda algoritama za pronalaženje MEM-ova. Ova metoda uspoređuje niz $Q$ veličine $m$ s indeksnom strukturom izgrađenom nad referentnim nizom $S$ veličine $n$. Prednost ovog pristupa je ta što se konstruirana indeksna struktura može ponovno iskoristiti, odnosno za sljedeću usporedbu nizova nije potrebno ponovno indeksirati referentni niz. 

U sljedećim poglavljima opisan je postupak izgradnje child polja te su prikazani rezultati usporedbe između izvedene implementacije i referentne implementacije.

\chapter{Child polje}
MEM-ovi se pronalaze obilaskom sufiksnog stabla odozgo prema dolje. Koristeći SA i LCP polje ovaj problem moguće je riješiti u složenosti $O(m + log n)$. Ukoliko se SA proširi dodatnim child poljem problem je moguće riješiti u složenosti $O(m)$.

Dohvat djece unutarnjih čvorova konceptualnog sufiksnog stabla ostvarenog pomoću SA i LCP polja moguće je obaviti u konstantnom vremenu koristeći child polje. Za izgradnju child polja koristi se informacija o LCP intervalima, definiranima u \citep{esa}. Bitno je naglasiti da su LCP intervali samo konceptualni te se nikada ne konstruiraju niti čuvaju u memoriji.

Child polje je duljine $n / K$, a svaki element u polju sadrži vrijednosti \textit{up}, \textit{down} i \textit{nextLIndex}. Te vrijednosti definirane su u nastavku:

$childtab[i].up = min\{q \in [0..i-1] \mid lcptab[q] > lcptab[i]$ i $\forall k \in [q+1..i-1] : lcptab[k] \geq lcptab[q]\}$

$childtab[i].down = max\{q \in [i+1..n] \mid lcptab[q] > lcptab[i]$ i $\forall k \in [i+1..q-1] : lcptab[k] > lcptab[q]\}$

$childtab[i].nextLIndex = min\{q \in [i+1..n] \mid lcptab[q] = lcptab[i]$ i $\forall k \in [i+1..q-1] : lcptab[k] > lcptab[i]\}$

Tablica \ref{tbl:example} prikazuje vrijednosti sufiks i LCP polja te sve tri vrijednosti child polja na primjeru \textit{S = mississippi}.

\begin{table}[h]
	\centering
	\caption{SA  niza \textit{S = mississippi} proširen s LCP i child tablicom.}
	\label{tbl:example}
	
	\begin{tabular}{ccccccl}
		\hline
    \texttt{i} & \texttt{sa[i]} & \texttt{lcp[i]} & \texttt{up[i]} & \texttt{down[i]} & \texttt{nextLIndex[i]} & \texttt{s[sa[i]]} \\ \hline
    0 & 10 &  0 &    &  1 &  2 & \texttt{i\$}           \\
    1 &  4 &  1 &    &    &    & \texttt{issippi\$}     \\
    2 &  0 &  0 &  1 &    &  3 & \texttt{mississippi\$} \\
    3 &  8 &  0 &    &    &  4 & \texttt{ppi\$}         \\
    4 &  6 &  0 &    &  5 &    & \texttt{sippi\$}       \\
    5 &  2 &  1 &    &    &    & \texttt{ssissippi\$}   \\ \hline
	\end{tabular}
\end{table}


Child polje nosi informaciju o odnosu roditelj-dijete u LCP intervalima. Za LCP interval [i..j] s \textit{l}-indeksima $i_1 < i _2 <$ ... $< i_k$, $childtab[i].down$ ili $childtab[j+1].up$ vrijednost koristi se za dobivanje prvog \textit{l}-indeksa $i_1$. Ostali \textit{l}-indeksi $i _2$, ..., $i_k$ dobivaju se redom iz $childtab[i_1].nextLIndex$, ..., $childtab[i_{k-1}].nextLIndex$. U \cite{esa} je dokazano da su sada child intervali LCP intervala [i..j] redom [$i..i_1-1$], [$i_1..i_2-1$], ..., [$i_k..j$].

\chapter{Konstrukcija child polja}

Algoritam \ref{alg:child-array} prikazuje izgradnju rijetkog child polja. Originalni algoritam dokazan je u \cite{esa}, a autori u \cite{essa} modificiraju ga za rad s rijetkim sufiksnim poljem. U poglavlju 2 pokazano je da je za izgradnju child polja potrebno imati samo informaciju sadržanu u LCP intervalima. Obzirom da uvođenje parametra rijetkosti $K$ ne utječe na definiciju LCP intervala, moguće je izgraditi rijetko child polje.

U prvom linearnom prolazu LCP polja računaju se \textit{up} i \textit{down} vrijednosti. Trenutni indeks stavlja se na vrh stoga ukoliko je njegova LCP-vrijednost veća ili jednaka LCP-vrijednosti s vrha stoga. U suprotnom, elementi se skidaju sa stoga dok su njihove LCP-vrijednosti veće od LCP-vrijednosti trenutnog indeksa. Usporedbom LCP-vrijednosti trenutnog indeksa i indeksa s vrha stoga, \textit{up} i \textit{down} vrijednostima se pridjeljuju elemenati skinutih sa stoga tijekom prolaska. \todo{dodati opis čišćenja stoga}

U drugom linearnom prolazu LCP polja računaju se \textit{nextLIndex} vrijednosti. Izračun \textit{nextLIndex} vrijednosti je znatno intuitivniji. Potrebno je usporediti LCP vrijednost trenutnog indeksa s LCP vrijednošću indeksa s vrha stoga. Ukoliko su navedene vrijednosti jednake, \textit{nextLValue} u child polju na indeksu s vrha stoga  postaje trenutni index.

\begin{algorithm}[h]
	\caption{Algoritam za konstrukciju child polja}
	\label{alg:child-array}

	\SetKwProg{Fun}{function}{ begin}{end}
	\SetKwFunction{cca}{constructChildArray}
	\SetKw{and}{and}
	
	\Fun{\cca{$LCP$}}{
	    $lastIndex = -1$\;
	    $ST.push(0)$\;
	    \For{$i = 1$ \KwTo $n / K - 1$}{
			\While{$LCP[i] < LCP[ST.top]$}{
			   $lastIndex=ST.pop$\;
			   \If{$LCP[i] \leq LCP[ST.top]$ \and $LCP[ST.top] \neq LCP[lastIndex]$}{
			       $CHILD[ST.top].down = lastIndex$\;
			   }
			}
			\If{$lastIndex \neq -1$}{
                $CHILD[i].up = lastIndex$\;
                $lastIndex = -1$\;			
			}
			$ST.push(i)$\;
		}
		\While{$0 < LCP[ST.top]$}{
            $lastIndex = ST.pop$\;
            \If{$0 \leq LCP[ST.top]$ \and $LCP[ST.top] \neq LCP[lastIndex]$}{
                $CHILD[ST.top].down = lastIndex$
            }		
		}
        \For{$i = 1$ \KwTo $n / K - 1$}{
            \While{$LCP[i] < LCP[ST.top]$}{
                $ST.pop$\;            
            }
            \If{$LCP[i] = LCP[ST.top]$}{
                $lastIndex = ST.pop$\;
                $CHILD[lastIndex].nextLIndex = i$\;
            }
            $ST.push(i)$\;
        }
	}
\end{algorithm}

\chapter{Rezultati}
U ovom poglavlju predstavljeni su rezultati implementacije child polja. \todo{napisati nesto kao za potrebe testiranja odabrane su dvije velike datoteke koje sadrzavaju genome bakterija (je li to tocno uopce?) i generirani su vlastiti primjeri.. -kako se testiralo? koji program se koristio za mjerenje memorije i vremena, testovi su izvršeni na računalu i operacijskom sustavu... od kud su preuzete ove datoteke?}

\section{Vrijeme izvođenja}
Prilikom traženja MEM-ova potrebno je paziti na vrijeme izvođenja. Kako se MEM-ovi traže nad velikim nizovima, rješenje problema prirodno teži k većem vremenu izvođenja. Potrebno je optimirati sve korake algoritma, tako i izgradnju child polja, kako bi se ostvarilo što kraće izvođenja. 

Tablice \ref{tbl:time-bacteria} i \ref{tbl:time-generated} prikazuju vrijeme konstrukcije child polja. Stupac $t_{impl}$  predstavlja vrijeme izvođenja izvedene implementacije dok stupac $t_{ref}$ predstavlja vrijeme izvođenja referentnog rješenja. Stupac $t_{razlika}$ predstavlja razliku u vremenu izvođenja između izvedene implementacije i referentnog rješenja. Vrijednosti stupca $t_{razlika}$ dobiveni su izrazom $(t_{ref} - t_{impl}) / t_{ref} * 100\%$.

Tablica \ref{tbl:time-bacteria} prikazuje usporedbu vremena izgradnje child polja nad postojećim genomima. Može se uočiti kako su vremena izvođenja izvedene implementacije i referentnog rješenja vrlo slična. Vrijeme izvođenja izvedenog rješenja je uglavnom manje od vremena izvođenja referentne implementacije za $K=5$ i $K=20$ dok za $K=100$ referentno rješenje ima kraće vrijeme izvođenja.

Tablica \ref{tbl:time-generated} prikazuje usporedbu vremena izgradnje child polja nad generiranim nizovima. Generirani nizovi su različitih broja baza. Vrijeme izvođenja izvedene implementacije i referentnog rješenja je vrlo slično te vrijede prethodni zaključci.
 
\begin{table}[h]
	\centering
	\caption{Usporedba vremena izvršavanja izvedene implementacije i referentnog rješenja za različite primjere i vrijednosti parametra $K$}
	\label{tbl:time-bacteria}
	
	\begin{tabular}{lrrrrrr}
		\hline
        Primjer & Broj baza & $K$ & $t_{impl}$ & $t_{ref}$ & $t_{razlika}$ \\ \hline
        \textit{E. coli} & 5676107 & 5 & 0.776514 & 0.861211 & 9.83\% \\
        \textit{E. coli} & 5676107 & 20 & 5.30142 & 5.3719 & 1.31\% \\
        \textit{E. coli} & 5676107 & 100 & 9.01133 & 8.96336 & -0.53\% \\ \hline
        \textit{E. coli} & 4393089 & 5 & 0.579196 & 0.697147 & 16.91\% \\
        \textit{E. coli} & 4393089 & 20 & 4.17962 & 4.21829 & 0.91\% \\
        \textit{E. coli} & 4393089 & 100 & 6.95086 & 6.83971 & -1.62\% \\ \hline
        \textit{Helicobacter} & 1617653 & 5 & 0.168793 & 0.198376 & 14.91\% \\
        \textit{Helicobacter} & 1617653 & 20 & 1.65821 & 1.59517 & -3.95\% \\
        \textit{Helicobacter} & 1617653 & 100 & 2.62 & 2.59599 & -0.92\% \\ \hline
        \textit{Methylobacillus} & 2971517 & 5 & 0.353354 & 0.389577 & 9.29\% \\
        \textit{Methylobacillus} & 2971517 & 20 & 2.95769 & 2.95796 & 0.00\% \\
        \textit{Methylobacillus} & 2971517 & 100 & 4.71582 & 4.66719 & -1.04\% \\
    \hline
	\end{tabular}
\end{table}


\begin{table}[h]
	\centering
	\caption{Usporedba vremena izvršavanja izvedene implementacije i referentnog rješenja za različite generirane primjere i različite vrijednosti parametra $K$}
	\label{tbl:time-generated}
	
	\begin{tabular}{lrrrrr}
		\hline
        Broj baza & $K$ & $t_{impl}$ & $t_{ref}$ & $t_{razlika}$ \\ \hline
        $10^5$ & 5 & 0.007086 & 0.014893 & 52.42\% \\
        $10^5$ & 20 & 0.124813 & 0.127244 & 1.91\% \\
        $10^5$ & 100 & 0.181825 & 0.15902 & -14.34\% \\ \hline
        $10^6$ & 5 & 0.106828 & 0.141809 & 24.66\% \\
        $10^6$ & 20 & 1.05831 & 1.04681 & -1.09\% \\
        $10^6$ & 100 & 1.59274 & 1.59819 & 0.034\% \\ \hline
        $10^7$ & 5 & 1.41559 & 1.56725 & 9.67\% \\
        $10^7$ & 20 & 9.81479 & 9.53324 & -2.95\% \\
        $10^7$ & 100 & 15.609 & 15.5091 & -0.064\% \\
    \hline
	\end{tabular}
\end{table}

\section{Potrošnja memorije}
Traženje MEM-ova nad genomima može dovesti do vrlo značajnih otkrića. Kako je dostupan sve veći broj cijelovitih genoma, algoritmi rade nad vrlo velikim nizovima. Zbog toga je potrebno voditi računa o efikasnosti korištenja memorije. 

Child polje je jedna od ključnih struktura podataka algoritma essaMEM i potrebno ju je konstruirati na efikasan način, odnosno da se ne troši više memorije no što je to zaista potrebno.

Tablice \ref{tbl:memory-bacteria} i \ref{tbl:memory-bacteria} prikazuju utrošak memorije child polja. Stupac $mem_{impl}$  predstavlja utrošak memorije izvedene implementacije dok stupac $mem_{ref}$ predstavlja utrošak memorije referentnog rješenja. Stupac $mem_{razlika}$ predstavlja razliku utrošene memorije između izvedene implementacije i referentnog rješenja. Vrijednosti stupca $mem_{razlika}$ dobiveni su izrazom $(mem_{ref} - mem_{impl}) / mem_{ref} * 100\%$.

Iz rezultata prikazanih tablicama \ref{tbl:memory-bacteria} i \ref{tbl:memory-generated} može se vidjeti kako izvedena implementacija ima manji utrošak memorije od referentnog rješenja za gotovo sve primjere.

\begin{table}[h]
	\centering
	\caption{Usporedba utrošene memorije izvedene implementacije i referentnog rješenja za različite primjere i vrijednosti parametra $K$}
	\label{tbl:memory-bacteria}
	
	\begin{tabular}{lrrrrrr}
		\hline
		Primjer & Broj baza & $K$ & $mem_{impl}$ & $mem_{ref}$ & $mem_{razlika}$ \\ \hline
        \textit{E. coli} & 5676107 & 5 & 24988 & 29652 & 15.72\% \\
        \textit{E. coli} & 5676107 & 20 & 11824 & 14004 & 15.56\% \\
        \textit{E. coli} & 5676107 & 100 & 9772 & 12908 & 24.29\% \\ \hline
        \textit{E. coli} & 4393089 & 5 & 21292 & 25108 & 15.19\% \\
        \textit{E. coli} & 4393089 & 20 & 11464 & 12016 & 4.59\% \\
        \textit{E. coli} & 4393089 & 100 & 11000 & 10180 & -8.05\% \\ \hline
        \textit{Helicobacter} & 1617653 & 5 & 7976 & 15416 & 48.26\% \\
        \textit{Helicobacter} & 1617653 & 20 & 4156 & 4752 & 12.54\% \\
        \textit{Helicobacter} & 1617653 & 100 & 3376 & 4760 & 29.07\% \\ \hline
        \textit{Methylobacillus} & 2971517 & 5 & 13752 & 20012 & 31.28\% \\
        \textit{Methylobacillus} & 2971517 & 20 & 6552 & 7312 & 10.39\% \\
        \textit{Methylobacillus} & 2971517 & 100 & 6456 & 8172 & 20.99\% \\
    \hline
	\end{tabular}
\end{table}

\begin{table}[h]
	\centering
	\caption{Usporedba utrošene memorije izvedene implementacije i referentnog rješenja za različite generirane primjere i različite vrijednosti parametra $K$}
	\label{tbl:memory-generated}
	
	\begin{tabular}{lrrrrr}
		\hline
        Broj baza & $K$ & $mem_{impl}$ & $mem_{ref}$ & $mem_{razlika}$ \\ \hline
        $10^5$ & 5 & 1880 & 10160 & 81.49\% \\
        $10^5$ & 20 & 1664 & 1832 & 9.17\% \\
        $10^5$ & 100 & 1736 & 1848 & 6.06\% \\ \hline
        $10^6$ & 5 & 5288 & 13208 & 59.96\% \\
        $10^6$ & 20 & 3100 & 3432 & 9.67\% \\
        $10^6$ & 100 & 2568 & 3436 & 25.26\% \\ \hline
        $10^7$ & 5 & 43268 & 44880 & 3.59\% \\
        $10^7$ & 20 & 20780 & 22264 & 6.66\% \\
        $10^7$ & 100 & 19076 & 21444 & 11.04\% \\
    \hline
	\end{tabular}
\end{table}

\section{Ovisnost resursa o parametrima}
\todo{komentirati grafove DANIEL} \\

\begin{figure}[!h]
	\centering
	\def\svgwidth{.7\columnwidth}
	\input{N-t.pdf_tex}
  \caption{Ovisnost vremena izvršavanja o broju baza ($K = 10$)}
\end{figure}

\begin{figure}[!h]
	\centering
	\def\svgwidth{.7\columnwidth}
	\input{N-mem.pdf_tex}
  \caption{Ovisnost utrošene memorije o broju baza ($K = 10$)}
\end{figure}

\begin{figure}[!h]
	\centering
	\def\svgwidth{.7\columnwidth}
	\input{K-t.pdf_tex}
  \caption{Ovisnost vremena izvršavanja o parametru $K$ ($N = 5 \cdot 10^6$)}
\end{figure}

\begin{figure}[!h]
	\centering
	\def\svgwidth{.7\columnwidth}
	\input{K-mem.pdf_tex}
  \caption{Ovisnost utrošene memorije o parametru $K$ ($N = 5 \cdot 10^6$)}
\end{figure}

\chapter{Zaključak}
\todo{napisati... DANIEL}
\todo{MEM je vrlo bitan alat u komparativnoj genomici jer omogucuje ... -sufiksna polja omogucuju stvaranje virtualnih sufiksnih stabala, ona su efikasnija od stabala, ali cijena toga je slozenost izgradnje... -napredak u realizaciji sufiksnih polja omogucuje izgradnju novih efikasnijih algoritama za pronalazak MEM... -essaMEM je predstavlja poboljsanje dosadasnjih algoritama, koristi slozen postupak pogodan za pronalazenje MEMova nad velikim nizovima ... -zbog svoje slozenosti javlja se pitanje reproducibilnosti... - implementirano rjesenje child polja postiže približno iste rezultate kao i referentno rjesenje odnosno u većini slucajeva i bolje ... - programsko rjesenje i dokumentacija dostupni su...}

\bibliography{literature}
\bibliographystyle{fer}

\begin{sazetak}
Komparativna genomika je područje bioinformatike u kojem se značajke genoma različitih organizama uspoređuju. Rastom broja dostupnih potpunih genoma organizama javlja se zahtjev za novim unaprijeđenim tehnikama usporedbe genoma. Bitan alat u komparativnoj genomici je traženje maksimalnih točnih slaganja. essaMEM je napredan algoritam za efikasno računanje maksimalnih točnih slaganja za velike nizove. Cilj ovog rada bio je proučiti algoritam essaMEM te implementirati izgradnju child polja. Na kraju rada predstavljeni su rezultati izgradnje child polja te su uspoređeni s referentnim rješenjem.

  \kljucnerijeci{MEM, enhanced sparse suffix array, child array, child array construction}
\end{sazetak}

\end{document}
